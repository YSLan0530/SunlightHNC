% This file is part of SunlightDPD - a home for open source software
% related to the dissipative particle dynamics (DPD) simulation
% method.

% Copyright (c) 2009-2017 Unilever UK Central Resources Ltd
% (Registered in England & Wales, Company No 29140; Registered
% Office: Unilever House, Blackfriars, London, EC4P 4BQ, UK).

% SunlightDPD is free software: you can redistribute it and/or
% modify it under the terms of the GNU General Public License as
% published by the Free Software Foundation, either version 3 of the
% License, or (at your option) any later version.

% SunlightDPD is distributed in the hope that it will be useful, but
% WITHOUT ANY WARRANTY; without even the implied warranty of
% MERCHANTABILITY or FITNESS FOR A PARTICULAR PURPOSE.  See the GNU
% General Public License for more details.

% You should have received a copy of the GNU General Public License
% along with SunlightDPD.  If not, see <http://www.gnu.org/licenses/>.

\documentclass[12pt,a4paper]{article}
\usepackage{graphicx}
\usepackage{amsmath}
\usepackage{amssymb}
\usepackage{color}

\DeclareMathOperator{\erf}{erf}
\DeclareMathOperator{\erfc}{erfc}
\DeclareMathOperator{\imaginary}{Im}

\newcommand{\latin}[1]{\emph{#1}}
\newcommand{\german}[1]{$\frak{#1}$}
\newcommand{\etal}{\latin{et al.}}
\newcommand{\etc}{\latin{et\,c.}}
\newcommand{\eg}{\latin{e.\,g.}}
\newcommand{\cf}{\latin{c.\,f.}}
\newcommand{\ie}{\latin{i.\,e.}}
\newcommand{\via}{\latin{via}}
\newcommand{\aposteriori}{\latin{a posteriori}}
\newcommand{\adhoc}{\latin{ad hoc}}
\newcommand{\viz}{\latin{viz.}}
\newcommand{\viceversa}{\latin{vice versa}}
\newcommand{\mutmut}{\latin{mutatis mutandis}}
\newcommand{\ansatz}{{ansatz}}
\newcommand{\perse}{\latin{per se}}

\newcommand{\half}{{\textstyle\frac{1}{2}}}

\newcommand{\alt}{\lesssim}
\newcommand{\agt}{\gtrsim}

\newcommand{\kB}{k_{\mathrm{B}}}
\newcommand{\kT}{\kB T}
\newcommand{\myex}{^{\mathrm{ex}}}
\newcommand{\Uex}{U\myex}
\newcommand{\Fex}{F\myex}
\newcommand{\muex}{\mu\myex}
\newcommand{\fNex}{f_N\myex}
\newcommand{\myvec}[1]{{\mathbf #1}}
\newcommand{\rvec}{\myvec{r}}
\newcommand{\kvec}{\myvec{k}}
\newcommand{\lr}{^{\mathrm{L}}}
\newcommand{\vlr}{v\lr}
\newcommand{\tildevlr}{{\tilde v}\lr}
\newcommand{\myprime}{^{{}\,\prime}}
%\newcommand{\sigmap}{{\overline\sigma}}
%\newcommand{\sigmap}{{\Sigma}}
\newcommand{\sigmap}{{\sigma'}{}}
\newcommand{\gc}{g^{c}}
\newcommand{\href}{h^{(0)}}

% The following are used to typeset a box of width character '0'
\newlength{\zerolen}
\settowidth{\zerolen}{0}
\newcommand{\zeroset}[1]{\makebox[\zerolen][c]{#1}}

\newcommand{\Eqref}[1]{Eq.~\eqref{#1}}
\newcommand{\Eqsref}[1]{Eqs.~\eqref{#1}}

\newcommand{\Refcite}[1]{Ref.~\cite{#1}}

\newcommand{\myav}[1]{\langle #1\rangle}

\newcommand{\FLAG}[1]{{\color{red} #1}}

\newcommand{\FORTRAN}{{\sc fortran}}
\newcommand{\python}{{\tt python}}
\newcommand{\LAPACK}{{\sc lapack}}
\newcommand{\FFTW}{{\sc fftw}}

\begin{document}

\noindent{\bf\large FORTRAN 90 Ornstein-Zernike solver --- version 1.8.1}\\[6pt]
\noindent{\it Patrick B Warren, Unilever R\&D Port Sunlight (January
  2017)}\\[18pt]

\noindent The code and this accompanying documentation is released
under the GPL.  You are welcome to use and modify it for your own
projects. If you end up publishing results based on this, please cite\\
%
\begin{center}
  \fbox{\parbox{0.9\textwidth}{P.~B.~Warren, A.~Vlasov, L.~Anton and
    A.~J.~Masters ``Screening properties of Gaussian electrolyte
    models, with application to dissipative particle dynamics'',
    J. Chem. Phys. {\bf138}, 204907 (2013).}}
\end{center}

\vspace{0.25in}
\noindent Features of version 1.8.1:
\begin{itemize}
\item FORTRAN 90 based, with example python driver scripts ;
\item HNC, MSA, RPA, and EXP closures ;
\item fast Ng solver, and Ng decomposition for electrostatics ;
\item multicomponent (arbitrary number of components) ;
\item hard core (RPM-like) and soft core (DPD-like) potentials ;
\item full structural thermodynamics ;
\item fully open source.
\end{itemize}


\noindent Added in version 1.8 and 1.8.1:
\begin{itemize}
\item MSA closure and EXP for hard cores (in version 1.8.1)
\end{itemize}

\noindent Added in version 1.7.1:
\begin{itemize}
\item exact and approximate pair potentials for Slater smearing. 
\end{itemize}

\newpage
\tableofcontents
\newpage

\section{Introduction}
%
The \FORTRAN\ 90 module \verb+oz_mod.f90+ implements an
Ornstein-Zernike solver for multicomponent mixtures of possibly
charged particles, returning both structural and thermodynamic
information.  The closures currently implemented are the HNC, MSA,
RPA, and EXP.  Though we say it ourselves, the code is \emph{fast}
since it is written in native \FORTRAN\ and implements the Ng
acceleration schemes \cite{Ng74}.  The interaction potentials are
specified in their own routines and the code could be used for other
potentials with minor modifications.  Further technical background to
liquid state integral equation methods can be found in Hansen and
McDonald \cite{HM06}, Kelley and Montgomery Pettitt \cite{KMP04}, and
Vrbka \etal\ \cite{Vrbka09}.  We note that Vrbka provides an
independent package called `pyOZ', implemented entirely in
\python~\cite{Vrbka09}.  There are some differences with the present
approach: the Ornstein-Zernike relation is normalised slightly
differently, a conjugate gradient method is used to accelerate
convergence, and pyOZ provides other closures in addition to the ones
provided here.

The present code is based on an HNC code developed by Lucian Anton,
modified for dissipative particle dynamics (DPD) by Andrey Vlasov with
help from Lucian and Andrew Masters.  The code was later converted by
Patrick Warren to use the open source libraries \FFTW\ and \LAPACK,
and rewritten to be compatible with the \FORTRAN-\python\ interface
generator \verb+f2py+.

\section{Theoretical background}
%
\subsection{Single component HNC}
%
We suppose that the interaction between a pair of particles is given
by the potential $v(r)$.  We introduce the following quantities
\cite{HM06}: the pair distribution function $g(r)$, the total
correlation function $h(r)\equiv g(r)-1$, the direct correlation
function $c(r)$ defined by the Ornstein-Zernike relation below, and
the indirect correlation function $e(r)\equiv h(r)-c(r)$.  Note that
the indirect correlation function is called $\gamma(r)$ by Vrbka
\etal\ \cite{Vrbka09} and $b(r)$ by Hansen and McDonald \cite{HM06}.  We
also introduce the Fourier transforms, \eg\
%
\begin{equation}
\tilde h(k) = \int\!d^3\rvec\, e^{-i\kvec\cdot\rvec} \,h(r)\,,
\end{equation}
%
which simplifies to the Fourier-Bessel transform
%
\begin{equation}
\tilde h(k) = \frac{4\pi}{k} \int_0^\infty \!\! dr\, \sin(kr)\, r\, h(r)\,,
\label{eq:fFB}
\end{equation}
%
and
%
\begin{equation}
h(r) = \int\! \frac{d^3\kvec}{(2\pi)^3} \,e^{i\kvec\cdot\rvec} \,\tilde h(k)\,,
\end{equation}
%
which simplifies to 
%
\begin{equation}
h(r) = \frac{1}{2 \pi^2 r} \int_0^\infty \!\!dk\, \sin(kr)\, k\, \tilde h(k)\,.
\label{eq:bFB}
\end{equation}
%
In terms of the Fourier transforms the Ornstein-Zernike (OZ) relation
is
%
\begin{equation}
\tilde h(k) = \tilde c(k) + \rho\,
\tilde h(k)\, \tilde c(k)
\label{eq:oz1a}
\end{equation}
%
where $\rho$ is the density.  This confirms the relevance of the
standard choice of normalisation of the 3d Fourier transform pair
which puts the factor $1/(2\pi)^3$ into the back transform.  The OZ
relation can be rearranged to 
%
\begin{equation}
\tilde h(k) = \frac{\tilde c(k)}{1-\rho \tilde c(k)}
\label{eq:oz1b}
\end{equation}
%
(\cf\ Eq.~(3.5.13) in \Refcite{HM06}).
The hyper-netted chain (HNC) closure is defined in real space and is
%
\begin{equation}
g(r)=\exp[-\beta v(r)+h(r)-c(r)]
\label{eq:hnc1a}
\end{equation}
%
(\cf\ Eq.~(4.3.19) in \Refcite{HM06}) where $\beta=1/\kT$.  It
amounts the neglect of the bridge diagrams.  For hard spheres HNC is
known to be inferior to Percus-Yevick, but for soft potentials like
DPD it generally gives excellent results.  In the presence of hard
cores, we note that $\exp[-\beta v]$ is still well defined (and zero,
in fact) where $\beta v$ is infinite.  We can accommodate hard cores
therefore by making sure that we always work with $\exp[-\beta v]$
instead of $\beta v$ in the numerics.  The function $\exp[-\beta v]$
can be pre-computed for a given potential.

To obtain the algorithm implemented in the code we rewrite
\Eqsref{eq:oz1b} and \eqref{eq:hnc1a} in terms of $c(r)$ and
$e(r)$.  The OZ relation becomes
%
\begin{equation}
\tilde e = \frac{\tilde c}{1-\rho\tilde c}-\tilde c
\label{eq:oz1c}
\end{equation}
%
and the HNC closure becomes
%
\begin{equation}
c=\exp[{-\beta v}+e]-e-1\,.
\label{eq:hnc1b}
\end{equation}
%
The algorithm is as follows.  We start with some guess for the direct
correlation function $c(r)$.  We Fourier transform this to $\tilde
c(k)$ and solve the OZ relation in \Eqref{eq:oz1c} to get the indirect
correlation function (in reciprocal space) $\tilde e(k)$.  We Fourier
back-transform to get $e(r)$ and plug this into the HNC closure in the
form of \Eqref{eq:hnc1b} to get a new direct correlation function
$c(r)$.  This cycle is iterated until $c(r)$ and $e(r)$ converge to a
self-consistent solution pair.  A direct approach like this is usually
numerically unstable so in the Picard scheme we mix a little of the
new $c(r)$ into the old $c(r)$, and iterate until we converge to a
solution.  In the Ng acceleration scheme we keep track of the
convergence trajectory and use this to accelerate convergence to the
solution.  The details of the Ng scheme will not be described here,
rather we point the interested reader to \Refcite{Ng74} and the code
itself.  The initial trajectory for the Ng scheme is generated by a
few Picard iterations.  Some possible initial choices for the direct
correlation function $c(r)$ are zero, $-\beta v(r)$ and $\exp[-\beta
  v(r)]-1$.  For soft potentials any of these will do in principle but
the initial rate of convergence may differ.  For hard core potentials,
the middle option cannot be used.
 
Now we describe Ng's splitting method for handling long range forces.
We partition the interaction potential into a short range part and a
long range part,
%
\begin{equation}
v(r)=v'(r)+v\lr(r)\,.
\end{equation}
%
It is generally accepted that $c(r)\sim-\beta v\lr(r)$ for
$r\to\infty$ so we make this explicit by partitioning both the direct
and indirect correlation functions,
%
\begin{equation}
c(r)=c'(r)-\beta v\lr(r)\,,\quad
e(r)=e'(r)+\beta v\lr(r)\,.
\end{equation}
%
These are taken to provide the definitions of the offset functions
$c'(r)$ and $e'(r)$.  We rewrite the OZ relation and the HNC closure
in terms of these,
%
\begin{equation}
\tilde e\myprime=\frac{\tilde c\myprime-\beta\tildevlr}{1-\rho(\tilde
  c\myprime-\beta\tildevlr)}-\tilde c\myprime
\label{eq:oz1d}
\end{equation}
and
\begin{equation}
c'=\exp[-\beta v'+e']-e'-1\,.
\label{eq:hnc1d}
\end{equation}
%
The first of these requires that we know the Fourier transform of the
long range part of the potential.  The main advantage of the Ng split
is that $c'(r)$ and $e'(r)$ are now genuinely short-ranged so the
numerical behaviour is much better, certain thermodynamic integrals
are guaranteed convergence, and the expected asymptotic behaviour of
$c(r)$ is explicitly incorporated.  The numerical solution scheme
based on \Eqsref{eq:oz1d} and \eqref{eq:hnc1d} goes through as before.
The initial choices for $c'(r)$ are replicated from above with $v$
replaced by $v'$.  If there is a hard core part of the potential, it
should be regarded as belonging to $v'(r)$, and we make sure we always
work with $\exp[-\beta v'(r)]$.

\subsection{Multicomponent HNC}
%
Now we turn to the multicomponent problem.  The pair functions
become for example $g_{\mu\nu}(r)$, \etc, and the problem is specified
by the interaction potential $v_{\mu\nu}(r)$ and the species densities
$\rho_\mu$.  The total density is $\rho=\sum_\mu \rho_\mu$ and the
species mole fractions are $x_\mu=\rho_\mu/\rho$.  We again split the
potential into short and long range contributions, 
%
\begin{equation}
v_{\mu\nu}(r)=v'_{\mu\nu}(r)+ v\lr_{\mu\nu}(r)
\end{equation}
%
where typically $v\lr_{\mu\nu}(r)$ incorporates the long range part of
the charge interactions.  Often the long range part satisfies a
symmetry condition (SYM) where
%
\begin{equation}
v\lr_{\mu\nu}(r)=z_\mu z_\nu v\lr(r)\quad\text{(SYM)}
\label{eq:sp}
\end{equation}
in which $z_\mu$ is the valency and $v\lr(r)$ is the interaction
potential between unit charges of the same sign.  However we will not
assume that $v\lr_{\mu\nu}(r)$ necessarily meets the SYM condition.  As
above we define $c'_{\mu\nu} = c_{\mu\nu} + \beta v\lr_{\mu\nu}$ and
$e'_{\mu\nu} = e_{\mu\nu} - \beta v\lr_{\mu\nu}$.  The multicomponent
HNC closure becomes
%
\begin{equation}
c_{\mu\nu}'=\exp[-\beta v_{\mu\nu}'+e_{\mu\nu}']-e_{\mu\nu}'-1\,.
\label{eq:hncnd}
\end{equation}
%
To derive the OZ relation in a usable form we start from 
the multicomponent analogue of \Eqref{eq:oz1a},
%
\begin{equation}
{\tilde h}_{\mu\nu} = {\tilde c}_{\mu\nu}+\rho\,{\textstyle\sum_\lambda}
\,x_\lambda\,{\tilde c}_{\mu\lambda}\,{\tilde h}_{\lambda\nu}\,.
\end{equation}
%
We write this as a matrix relation, introducing $R$ as a diagonal
matrix with entries $\rho_\mu=\rho x_\mu$,
%
\begin{equation}
{\tilde H}={\tilde C}+{\tilde C}\cdot R\cdot{\tilde H}\,.
\label{eq:oz2}
\end{equation}
%
Note that all the matrices in this are symmetric, so we can equally
well write this as
%
\begin{equation}
{\tilde H}={\tilde C}+{\tilde H}\cdot R\cdot{\tilde C}\,.
\label{eq:oz2a}
\end{equation}
%
Introducing next ${\tilde C}' = {\tilde C} + \beta{\tilde U}\lr$ and
${\tilde E}'={\tilde E}-\beta{\tilde U}\lr$ in \Eqref{eq:oz2}, and
rearranging, gives eventually
%
\begin{equation}
{\tilde E}'=[I-({\tilde C}'-\beta{\tilde U}\lr)\cdot R]^{-1}\cdot
[({\tilde C}'-\beta{\tilde U}\lr)\cdot R\cdot {\tilde C}'-\beta{\tilde
    U}\lr]\,.
\label{eq:oznd}
\end{equation}
%
The solution scheme is essentially as before.  Given an initial guess
for $c_{\mu\nu}'(r)$ we Fourier transform, evaluate the matrix
expression in \Eqref{eq:oznd}, and Fourier back-transform to obtain
$e_{\mu\nu}'(r)$.  We substitute this into the HNC closure in
\Eqref{eq:hncnd} to obtain a new guess for $c_{\mu\nu}'(r)$.  The
cycle is iterated to convergence and in practice a few Picard
iterations are used to pump-prime a multicomponent version of the Ng
acceleration scheme.  As before, any hard core part of the potential
should be regarded as belonging to $\beta v_{\mu\nu}'$ and the
numerical scheme should be constructed to use only $\exp[-\beta
  v_{\mu\nu}']$ (which can be pre-computed).

\subsubsection{Pair functions}
%
Given a converged solution pair $(c_{\mu\nu}', e_{\mu\nu}')$, the
total correlation functions can be evaluated from
%
\begin{equation}
h_{\mu\nu}(r) = c'_{\mu\nu}(r) +  e'_{\mu\nu}(r)\label{eq:hrs}
\end{equation}
%
(note that $\beta v_{\mu\nu}\lr$ cancels).  The pair functions are
given by $g_{\mu\nu}=\delta_{\mu\nu}+h_{\mu\nu}$.

\subsubsection{Structure factors}
%
For the partial structure factors, there is a choice of normalisation.
In the present code the structure factors are perhaps unusually
defined to be
%
\begin{equation}
S_{\mu\nu}(k) = \rho_\mu\delta_{\mu\nu} + \rho_\mu\rho_\nu{\tilde
  h}_{\mu\nu}
\label{eq:skdef}
\end{equation}
%
where ${\tilde h}_{\mu\nu}={\tilde c}_{\mu\nu}\myprime + {\tilde
  e}_{\mu\nu}\myprime$ (again, $\beta v_{\mu\nu}\lr$ cancels); the
Fourier transforms ${\tilde c}_{\mu\nu}\myprime$ and ${\tilde
  e}_{\mu\nu}\myprime$ are available as a byproduct of solving the OZ
relation.  The unusual choice of normalisation is made to facilitate
the calculation of the density-density and charge-charge structure
factors, \viz\
%
\begin{equation}
  S_{NN}(k) = \frac{\sum_{\mu\nu}S_{\mu\nu}(k)}{\sum_\mu\rho_\mu}\,,\quad
  S_{ZZ}(k) = \frac{\sum_{\mu\nu}z_\mu z_\nu S_{\mu\nu}(k)}{\sum_\mu\rho_\mu}
  \label{eq:snnszz}
\end{equation}
%
(assuming the ion valencies are $z_\mu$ for the latter).
With the above normalisation the structure factors obey
$S_{\mu\nu}(k)\to\rho_\mu\delta_{\mu\nu}$ as $k\to\infty$.  An
alternative (and more popular) normalisation, for which
$S_{\mu\nu}(k)\to\delta_{\mu\nu}$ as $k\to\infty$, is obtained from
the above expression by multiplying by $({\rho_\mu\rho_\nu})^{-1/2}$.

\subsection{Multicomponent MSA}
%
The closure in the mean spherical approximation (MSA) is
$g_{\mu\nu}(r)=0$ for $r\le d_{\mu\nu}$, and $c_{\mu\nu}(r)= -\beta
v_{\mu\nu}(r)$ for $r> d_{\mu\nu}$.  Here, $d_{\mu\nu}$ is the range
of the hard core repulsion between species $\mu$ and $\nu$
(\ie\ $v'_{\mu\nu}=\infty$ for $r\le d_{\mu\nu}$).  Rewriting as usual
in terms of our offset functions
$v'_{\mu\nu}=v_{\mu\nu}-v\lr_{\mu\nu}$, $c'_{\mu\nu}=c_{\mu\nu}+\beta
v\lr_{\mu\nu}$ and $e'_{\mu\nu}=e_{\mu\nu}-\beta v\lr_{\mu\nu}$, where
$h_{\mu\nu}=g_{\mu\nu}-1$ and $e_{\mu\nu}=h_{\mu\nu}-c_{\mu\nu}$, the
MSA closure is
%
\begin{equation}
  c'_{\mu\nu}=\Bigl\{\begin{array}{ll}
  -e'_{\mu\nu}-1 & (r<d_{\mu\nu})\,,\\[3pt]
  -\beta v'_{\mu\nu} & (r>d_{\mu\nu})\,.
  \end{array}
\end{equation}
%
This is almost a drop-in replacement for the HNC closure in
\Eqref{eq:hncnd}.  For a cold start we can initialise $c'_{\mu\nu}=-1$
for $r\le d_{\mu\nu}$.  In practice we can leave $c'_{\mu\nu}=-\beta
v'_{\mu\nu}$ untouched for $r>d_{\mu\nu}$ during the iterative
solution.  However, we make sure that whenever the potential is
changed $c'_{\mu\nu}$ is correctly reset.

\subsection{Multicomponent RPA}\label{sec:RPA}
%
The RPA closure is a special case of the MSA when there are no hard
cores, and is represented by $c_{\mu\nu}(r)=-\beta v_{\mu\nu}(r)$.
This is in fact one of the choices for initialising the HNC solver.
Given the HNC machinery, the implementation of the RPA is almost
completely trivial and comprises a single round trip through the OZ
solver starting from $c'_{\mu\nu}(r)=-\beta v'_{\mu\nu}(r)$.

\subsection{Multicomponent EXP}\label{sec:EXP}
%
The EXP approximation is a  refinement of the MSA/RPA
in which the  total correlation functions are replaced by
%
\begin{equation}
  h_{\mu\nu}\to(1+\href_{\mu\nu})\exp(h_{\mu\nu}-\href_{\mu\nu})-1
  \label{eq:exp}
\end{equation}
%
in which $\href_{\mu\nu}$ will be a hard sphere reference fluid in
the case of MSA, and $\href_{\mu\nu}=0$ in the case of RPA (no hard
cores).  The EXP breaks unwarranted symmetries such as
$h_{++}=-h_{+-}$ and ensures that $h_{\mu\nu}(r)>-1$ which is not
always satisfied by the MSA/RPA.  In practice EXP can be nearly as
good as HNC and extends to state points where HNC doesn't have a
solution.

Since the EXP approximation is defined as an action on the correlation
functions in real space, a round trip through the OZ relation is
required to compute the corresponding offset direct and indirect
correlation functions.  To do this we rewrite the OZ relation in
\Eqref{eq:oz2a} as
%
\begin{equation}
{\tilde C} = (I+{\tilde H}\cdot R)^{-1}\cdot{\tilde H}\,.
\label{eq:ozndb}
\end{equation}
%
With this implemented, $c'_{\mu\nu}=c_{\mu\nu}+\beta
v\lr_{\mu\nu}$ and $e'_{\mu\nu}= h_{\mu\nu} - c'_{\mu\nu}$ follow.

\subsection{Thermodynamics}
\label{sec:thermo}
%
The above section completely specifies the various closures for the
integral equation problem for multicomponent system.  We provide here
the suite of thermodynamic quantities which can be evaluated from the
converged solution pair $(c_{\mu\nu}', e_{\mu\nu}')$.  Some care is
needed in the case of hard core potentials, where $v_{\mu\nu}=\infty$
for $r<d_{\mu\nu}$.

The first thermodynamic quantity is the so-called virial route
compressibility factor
%
\begin{equation}
\frac{\beta p}{\rho}=1-\frac{2\pi}{3}\,{\textstyle \sum_{\mu\nu}}\,
\rho x_\mu x_\nu \int_0^\infty\!\!dr\,r^3\,\frac{\partial(\beta
  v_{\mu\nu})}{\partial r} g_{\mu\nu}(r)
\end{equation}
%
(\cf\ Eq.~(1.2) in \Refcite{Vrbka09}).  To handle the hard core
contribution to this we follow the line of argument presented in
Hansen and McDonald \cite{HM06} and introduce the function
$y_{\mu\nu}=\exp[\beta v_{\mu\nu}]\,g_{\mu\nu}$, which is continuous
through into the hard core region.  The above integral can be written
%
\begin{equation}
\begin{split}
I_{\mu\nu}&=\int_0^\infty\!\!dr\,r^3\,\frac{\partial(\beta
  v_{\mu\nu})}{\partial r} \exp[-\beta v_{\mu\nu}]\,y_{\mu\nu}(r)\\[6pt]
&{}\hspace{8em}{}=-\int_0^\infty\!\!dr\,r^3\,\frac{\partial(\exp[-\beta
    v_{\mu\nu}])}{\partial r} \,y_{\mu\nu}(r)\,.
\end{split}
\end{equation}
%
Let us write
%
\begin{equation}
\exp[-\beta v_{\mu\nu}]=\Theta(r-d_{\mu\nu})\,\exp[-\beta \overline v_{\mu\nu}]
\end{equation}
%
where $\overline v_{\mu\nu}=v_{\mu\nu}$ for $r\ge d_{\mu\nu}$, and we
don't care what it does for $r<d_{\mu\nu}$ except that it should be
\emph{continuous} through $r=d_{\mu\nu}$.  The Heaviside step function
$\Theta(x)=0$ for $x<0$ and $\Theta(x)=1$ for $x\ge0$, with derivative
$\Theta'(x)=\delta(x)$. Then
\begin{equation}
\begin{split}
I_{\mu\nu}&=-\int_0^\infty\!\!dr\,r^3\,\Bigl(\delta(r-d_{\mu\nu})
\exp[-\beta \overline v_{\mu\nu}])\\[3pt]
&{}\hspace{8em}{}+\Theta(r-d_{\mu\nu})
\frac{\partial(\exp[-\beta
    \overline v_{\mu\nu}])}{\partial r} \Bigr)\,y_{\mu\nu}(r)\\[9pt]
&=-g_{\mu\nu}(d_{\mu\nu})\,d_{\mu\nu}^3+
\int_{d_{\mu\nu}}^\infty\!\!dr\,r^3\,\frac{\partial(\beta
  v_{\mu\nu})}{\partial r} g_{\mu\nu}(r)
\end{split}
\end{equation}
%
In the second line we have restored $v_{\mu\nu}$ and $g_{\mu\nu}$. 
Thus, finally,
%
\begin{equation}
\frac{\beta p}{\rho}=1
+\frac{2\pi}{3}\,{\textstyle \sum_{\mu\nu}}\,\rho x_\mu x_\nu 
\Bigl[g_{\mu\nu}(d_{\mu\nu})\,d_{\mu\nu}^3 
- \int_{d_{\mu\nu}}^\infty\!\!dr\,r^3\,\frac{\partial(\beta
  v_{\mu\nu})}{\partial r} g_{\mu\nu}(r)\Bigr]
\end{equation}
%
This now contains explicitly the famous contact contribution to the
pressure, and the integral is now restricted to the region external to
the hard core.  In the case of a soft potential (DPD, URPM), the
contact contribution vanishes and the integral extends to $r=0$.

In practice we write the integral contribution to the virial pressure as
%
\begin{equation}
-\frac{2\pi}{3}\int_{d_{\mu\nu}}^\infty\!\!dr\,r^3\,
\frac{\partial({\textstyle \sum_{\mu\nu}}\,
\rho x_\mu x_\nu\,\beta
  v_{\mu\nu})}{\partial r}
+{\textstyle \sum_{\mu\nu}}\,
\rho x_\mu x_\nu t^{\mathrm{V}}_{\mu\nu}
\label{eq:vrpb}
\end{equation}
%
where 
%
\begin{equation}
t^{\mathrm{V}}_{\mu\nu}=-\frac{2\pi}{3}\int_{d_{\mu\nu}}^\infty\!\!dr\,r^3\,
\frac{\partial(\beta v'_{\mu\nu}+\beta v_{\mu\nu}\lr)}{\partial r} 
\,(c'_{\mu\nu}+e'_{\mu\nu})
\end{equation}
%
The first term in \Eqref{eq:vrpb} is the result for $g_{\mu\nu}=1$ and
is the mean-field contribution.  For many applications it is important
to evaluate this analytically, as the individual contributions may
diverge.  Under the SYM condition the contribution from
$v_{\mu\nu}\lr$ to the mean field term vanishes since charge
neutrality implies $\sum_{\mu\nu}\rho x_\mu x_\nu v_{\mu\nu}\lr
\propto v\lr(\sum_\mu \rho_\mu z_\mu)^2 = 0$.  The second term in
\Eqref{eq:vrpb} is the correlation correction (note
$c'_{\mu\nu}+e'_{\mu\nu}=g_{\mu\nu}-1$).

Next up is the compressibility itself which we write as
%
\begin{equation}
\frac{\partial(\beta p)}{\partial\rho}=
1-{\textstyle \sum_{\mu\nu}}\,\rho x_\mu x_\nu t^{\mathrm{C}}_{\mu\nu}
\end{equation}
%
where
%
\begin{equation}
t^{\mathrm{C}}_{\mu\nu}={4\pi}\int_0^\infty\!\!dr\,r^2\,(c_{\mu\nu}'(r)
-\beta v_{\mu\nu}\lr)
\end{equation}
%
(\cf\ Eq. (18) in \Refcite{Vrbka09}; the \emph{compressibility} is
not to be confused with the above \emph{compressibility factor}), In
terms of this, the isothermal compressibility is
$\chi_T=[\rho(\partial p/\partial\rho)]^{-1}$.  The contribution from
$v_{\mu\nu}\lr$ can also often be evaluated analytically and vanishes
under the SYM condition by the same argument as above.  Obviously the
compressibility can be integrated along an isotherm to obtain an
alternative expression for the pressure.  This is the so-called
compressibility route to the equation of state.  This expression is
insensitive to the presence of hard cores since they are supposed to
be contained in $v_{\mu\nu}'$ and enter \via\ the direct correlation
function. 

The next quantity of interest is the internal energy $U$.  Per
particle, the excess internal energy is
%
\begin{equation}
\frac{\Uex}{N}=2\pi\,{\textstyle \sum_{\mu\nu}}\,\rho x_\mu x_\nu 
\int_{d_{\mu\nu}}^\infty\!\!dr\, r^2\, v_{\mu\nu}(r)\,g_{\mu\nu}(r)
\end{equation}
%
(\cf\ Eq. (2.5.20) in \Refcite{HM06}).  Again this can be split into
a mean field term and a correlation correction,
%
\begin{equation}
\frac{\beta\Uex}{N}={2\pi}\int_{d_{\mu\nu}}^\infty\!\!dr\,r^2\,
({\textstyle \sum_{\mu\nu}}\,\rho x_\mu x_\nu\,\beta
  v_{\mu\nu})
+{\textstyle \sum_{\mu\nu}}\,
\rho x_\mu x_\nu t^{\mathrm{E}}_{\mu\nu}
\end{equation}
%
where 
%
\begin{equation}
t^{\mathrm{E}}_{\mu\nu}={2\pi}\int_{d_{\mu\nu}}^\infty\!\!dr\,r^2\,
(\beta v'_{\mu\nu}+\beta v_{\mu\nu}\lr)\,(c'_{\mu\nu}+e'_{\mu\nu})\,.
\end{equation}
%
An integration by parts shows that the mean field term here is related
to that for the compressibility factor,
%
\begin{equation}
{2\pi}\int_{d_{\mu\nu}}^\infty\!\!dr\,r^2\,
\beta v_{\mu\nu}
=
-\frac{2\pi\beta v_{\mu\nu}(d_{\mu\nu})\,d_{\mu\nu}^3}{3}
-\frac{2\pi}{3}\int_{d_{\mu\nu}}^\infty\!\!dr\,r^3\,
\frac{\partial(\beta
  v_{\mu\nu})}{\partial r}\,.\label{eq:same}
\end{equation}
%
The excess internal energy density (per unit volume) is
$\Uex/V=\rho\times (\Uex/N)$.  The equation of state can be derived
from the internal energy and this is the so-called energy route.

Last, and {\bf only valid for the HNC closure}, are thermodynamic
integrals for excess chemical potentials.
The relevant result is
%
\begin{equation}
\beta\muex_\mu = 4\pi\,{\textstyle \sum_{\nu}}\,\rho x_\nu 
\int_0^\infty\!\!dr\, r^2\, 
    [\half\,h_{\mu\nu}(r)\,e_{\mu\nu}(r)-c_{\mu\nu}(r)]\quad
    \Bigl(\>{}\equiv\ln\gamma_\mu\>\Bigr)
\end{equation}
%
(\cf\ section II\,C in \Refcite{Vrbka09}).  We
write this result as $\beta\muex_\mu = {\textstyle \sum_{\nu}}\,\rho x_\nu
t^{\mathrm{M}}_{\mu\nu}$ where
%
\begin{equation}
t^{\mathrm{M}}_{\mu\nu}=
4\pi \int_0^\infty\!\!dr\, r^2\, 
[\half\,(c'_{\mu\nu}+e'_{\mu\nu})(e'_{\mu\nu}+\beta v_{\mu\nu}\lr)
-c'_{\mu\nu}+\beta v_{\mu\nu}\lr]\,.
\end{equation}
%
The contribution from $v_{\mu\nu}\lr$ (the last term) can usually be
evaluated analytically and is the same as that appearing in the
compressibility (to within an overall sign).  These expressions, valid
only for HNC, offer a fourth route to the equation of state (the
chemical-potential route).  These expressions are insensitive to the
presence of hard cores.

In HNC both the virial route pressure and the chemical potentials
correspond to a certain free energy.  Since they are compatible in
this way, the excess free energy density follows from
%
\begin{equation}
{\beta\Fex}/{V} = \beta{\textstyle \sum_{\mu}}\,\rho x_\mu \muex_\mu
-\beta p + \rho
\end{equation}
%
where the pressure is calculated from the virial route compressibility
factor.

\section{Potentials}
%
\subsection{DPD potential}
\label{sec:dpd}
%
The HNC code contains routines which specify a multicomponent
potential for dissipative particle dynamics (DPD) with Gaussian
charges.  The potential in this case consists of short range and long
range contributions which can be mapped exactly to the Ng split.  The
short range part is
%
\newcommand{\rc}{r_c}
\newcommand{\lB}{l_{\mathrm{B}}}
\begin{equation}
v'_{\mu\nu}(r)=\left\{\begin{array}{ll}
\frac{1}{2}A_{\mu\nu}\kT(1-r/\rc)^2 & r<\rc\\[3pt]
0 & r \ge \rc
\end{array}\right.\,.
\end{equation}
%
This is a conventional choice, depending on a dimensionless symmetric
repulsion amplitude matrix $A_{\mu\nu}$ and cut off beyond a
distance $\rc$.  

Unlike the short range part there is no consensus on the best choice
for the long range interaction although the differences can always be
adsorbed into a redefinition of the short range potential.  The first
choice of Gaussian charges can be supported for a range of practical
reasons \cite{WVA+13, WV14}.  The Gaussian charges have size
$\sim\sigma$ and a density distribution $(2\pi\sigma^2)^{-3/2}
\exp({-r^2/2\sigma^2})$.  The corresponding interaction satisfies the
SYM condition and is given by
%
\begin{equation}
v_{\mu\nu}\lr(r)=z_\mu z_\nu v\lr(r)\,,\quad
v\lr(r)=\frac{\lB\kT}{r}
\erf\Bigl(\frac{r}{2\sigma}\Bigr)
\label{eq:dpdvlr}
\end{equation}
%
where $\lB$ is the coupling strength (the Bjerrum length).  The
precise definition of $\sigma$ is made to simplify the Fourier
transform (see below).

For use with the above expressions, we need the derivative of both
parts of the potential, and the Fourier transform of the long range
part.  These are
%
\begin{equation}
\frac{\partial(\beta v_{\mu\nu}')}{\partial r}=
\left\{\begin{array}{ll}
-(A_{\mu\nu}/\rc)(1-r/\rc) & r<\rc\\[3pt]
0 & r \ge \rc
\end{array}\right.
\end{equation}
%
\begin{equation}
\frac{\partial(\beta v\lr)}{\partial r}=
-\frac{\lB}{r^2}
\erf\Bigl(\frac{r}{2\sigma}\Bigr)
+\frac{\lB}{r\sigma\sqrt\pi}\,
e^{-r^2/4\sigma^2}\,,
\end{equation}
%
and
%
\begin{equation}
\beta {\tilde v}\lr(k)=\frac{4\pi\lB}{k^2}\times e^{-k^2\sigma^2}\,.
\end{equation}
%
The mean field contributions to the virial route pressure and energy
are given by (see also \Eqref{eq:same})
%
\begin{equation}
{2\pi}\int_0^\infty\!\!dr\,r^2\,
({\textstyle \sum_{\mu\nu}}\,\rho x_\mu x_\nu\,\beta
  v_{\mu\nu})=\frac{\pi\rc^3}{30}\,{\textstyle
  \sum_{\mu\nu}}\,
\rho x_\mu x_\nu A_{\mu\nu}\,.
\end{equation}
%
Because of the SYM condition, the contribution from $v\lr$ vanishes
from this, and also vanishes from the mean field contributions to the
compressibility and chemical potentials (this is true for all the
cases in this section).

This DPD model depends on the dimensionless repulsion amplitude matrix
$A_{\mu\nu}$, the ratios $\lB/\sigma$ and $\sigma/\rc$, and the
dimensionless density $\rho r_c^3$ \cite{WVA+13}.  In these terms
$\rc=1$ is often used as the fundamental length scale.  The ratio
$\lB/\sigma$ plays the role of an inverse effective temperature for
the Coloumbic part of the potential.

The above is the Gaussian charge case and is the default.  If Bessel
charges are selected the charge distribution becomes $K_1(r/\sigma) /
2\pi^2\sigma^2 r$ and the expressions change to \cite{WV14}
%
\begin{equation}
\beta v\lr(r)=\frac{\lB}{r}(1-e^{-r/\sigma})\,,
\end{equation}
%
\begin{equation}
\frac{\partial(\beta v\lr)}{\partial r}=
-\frac{\lB}{r^2}\Bigl[1-e^{-r/\sigma}\Bigl(1+\frac{r}{\sigma}\Bigr)\Bigr]\,,
\end{equation}
%
\begin{equation}
\beta {\tilde v}\lr(k)=\frac{4\pi\lB}{k^2}\times\frac{1}{1+k^2\sigma^2}\,.
\end{equation}
%
Note that $\sigma$ here is chosen to match the second moment of the
charge distribution for the Gaussian case.

Another alternative is linear charge smearing proposed by Groot
\cite{Groot03}.  In this case the charge distribution is $(3/\pi
R^3)(1-r/R)$ for $r<R$, vanishing for $r>R$. This gives rise to an
interaction potential in reciprocal space \cite{WV14}
%
\begin{equation}
\beta {\tilde v}\lr(k)=\frac{4\pi\lB}{k^2}
\Bigl(\frac{24-24\cos kR-12kR\sin kR}{k^4R^4}\Bigr)^2\,.
\end{equation}
%
A closed form expression in real space is not available, hence is not
implemented in the code.  This means that the thermodynamics is
\emph{not available} in this case, though the HNC solution goes
through since this only requires the reciprocal space potential.  The
term in large brackets is the charge density in reciprocal space.
Matching the second moment of the charge distribution \cite{WV14}
shows that the equivalent Gaussian smearing length is
$\sigma=R\sqrt{2/15}$.

Another popular alternative is Slater smearing proposed by
Gonz\'alez-Melchor \etal\ \cite{GM+06}.  They use an approximate
expression for the interaction potential, for which we have
%
\begin{equation}
  \beta v\lr(r)=\frac{\lB}{r}\Bigl[1-e^{-2\beta r}(1+\beta r)\Bigr]\,.
  \label{eq:slaterapprox}
\end{equation}
%
\begin{equation}
  \frac{\partial(\beta v\lr)}{\partial r}=
  -\frac{\lB}{r^2}\Bigl[1-e^{-2\beta r}(1+2\beta r(1+\beta r))\Bigr]\,,
\end{equation}
%
\begin{equation}
\beta {\tilde v}\lr(k)=\frac{4\pi\lB}{k^2}\times\frac{1}{(1+k^2/4\beta^2)^2}\,.
\end{equation}
%
These actually corresponds to a charge density $\beta^2 e^{-2\beta
  r}/\pi r$ ($\beta$ is a parameter here, not to be confused with
$1/\kT$). The equivalent Gaussian smearing length is here
$\sigma=1/\beta\sqrt{2}$.

The intended charge density in Slater smearing is
$(1/\pi\lambda^3)e^{-2r/\lambda}$.  It turns out that there are exact
expressions for the interaction between such charges \cite{WV14},
namely
%
\begin{equation}
  \beta v\lr(r)=\frac{\lB}{r}\Bigl[1-e^{-2r/\lambda}\Bigl(
    1+\frac{11r}{8\lambda}+\frac{3r^2}{4\lambda^2}
    +\frac{r^3}{6\lambda^3}\Bigr)\Bigr]\,,\label{eq:slaterexact}
\end{equation}
%
\begin{equation}
  \frac{\partial(\beta v\lr)}{\partial r}=
  -\frac{\lB}{r^2}\Bigl[1-e^{-2r/\lambda}\Bigl(
    1+\frac{2r}{\lambda}+\frac{2r^2}{\lambda^2}
    +\frac{7r^3}{6\lambda^3}+\frac{r^4}{3\lambda^4}\Bigr)\Bigr]\,,
\end{equation}
%
\begin{equation}
\beta {\tilde v}\lr(k)=\frac{4\pi\lB}{k^2}\times\frac{1}{(1+k^2\lambda^2/4)^4}\,.
\end{equation}
%
The equivalent Gaussian smearing length for these exact expressions is
$\sigma=\lambda$.  Originally, Gon\-z\'a\-lez-Melchor
\etal\ \cite{GM+06} had $\beta=1/\lambda$ in \Eqref{eq:slaterapprox},
but choosing $\beta=5/(8\lambda)$ makes \Eqsref{eq:slaterapprox} and
\eqref{eq:slaterexact} have the same limiting value as $r\to0$ and
brings the approximate interaction much closer to the exact result.
Note that in terms of the reciprocal space potential $\beta {\tilde
  v}\lr(k)=4\pi\lB k^{-2}(1+\sigma^2k^2/n)^{-n}$, there is a natural
hierarchy from Bessel ($n=1$), to approximate Slater ($n=2$), to
exact Slater ($n=4$), to Gaussian ($n\to\infty$).

\subsection{Softened URPM potential}
\label{sec:softURPM}
%
Another potential provided by the code at present is for the URPM in
which the unlike pair potential is artificially softened.  The
standard URPM (an equimolar mixture of Gaussian charges) is given by
the DPD potential above with $z_\mu=\pm1$ and $A_{\mu\nu}=0$.  There
are two ways the softened version can be implemented.  The first way
is to work purely in reciprocal space.  It is important to note that
this takes the model away from the SYM condition expressed in
\Eqref{eq:sp}.  In this approach the short range part $v'_{\mu\nu}=0$,
and the long range part is
%
\begin{equation}
v_{11}\lr=v_{22}\lr=\frac{\lB\kT}{r}
\erf\Bigl(\frac{r}{2\sigma}\Bigr)\quad
v_{12}\lr=-\frac{\lB\kT}{r}
\erf\Bigl(\frac{r}{2\sigmap}\Bigr)
\end{equation}
%
where typically $\sigmap>\sigma$ (in the case $\sigmap=\sigma$ we have
the original URPM which is also contained in the DPD potential
described above).  The potential in reciprocal space and the
derivatives follow \mutmut\ from the DPD case.  In the alternative
approach we retain the SYM condition so that the long range part is
given by \Eqref{eq:dpdvlr} and the short range part is
%
\begin{equation}
v'_{12}\equiv-\Delta v_{12}=-\frac{\lB\kT}{r}
\erf\Bigl(\frac{r}{2\sigmap}\Bigr)
+\frac{\lB\kT}{r}
\erf\Bigl(\frac{r}{2\sigma}\Bigr)
\label{eq:dv12}
\end{equation}
%
(obviously, $v'_{11}=v'_{22}=0$).  We define $\Delta v_{12}$ as the
potential that should be \emph{added} to the softened URPM, to recover
the standard URPM.  This is so that we can use the softened version as
a reference fluid.

For both routes the mean field contributions to the virial route
pressure and energy are non-zero only for the unlike pairs.  We can
evaluate them with $v'_{12}$ given by \Eqref{eq:dv12}
%
\begin{equation}
-\frac{2\pi}{3}\int_0^\infty\!\!dr\,r^3\,
\frac{\partial(\beta
  v'_{12})}{\partial r}={2\pi}\int_0^\infty\!\!dr\,r^2\,
\beta  v'_{12}=2\pi\lB(\sigmap^2-\sigma^2)\,.
\end{equation}
%
If we take the first, purely reciprocal space route, the failure to
satisfy the SYM condition means the compressibility and chemical
potentials need to take account of $v_{\mu\nu}\lr$.  This leads to the
long range contributions 
%
\begin{equation}
{4\pi}\int_0^\infty\!\!dr\,r^2\,\beta
v'_{12}=4\pi\lB(\sigmap^2-\sigma^2)\,.
\end{equation}
%
These should not be included if we follow the second route.

\subsection{RPM and softened RPM potential}
\label{sec:softRPM}
%
Another potential which can be solved in the code is for charged hard
spheres: both the `vanilla' RPM, and a version in which the unlike
pair potential is artificially softened, are provided.  For charged
hard spheres the interaction potential $v'_{\mu\nu}(r)=\infty$ for
$r<\sigma$ where $\sigma$ is the hard core diameter.  This is regarded
as belonging to the short range interaction potential, for which we
are always careful to use the exponentiated version.  Thus hard cores
are actually implemented by calculating $\exp[-\beta v'_{\mu\nu}]$
using one of the choices below, and setting this to zero for
$r<\sigma$.

Given this, we might na\"\i{}vely think of using $\lB/r$ as the long
range part of the potential.  Although the divergence at $r\to0$ is
hidden inside the hard core, this choice forces the discretised offset
functions $c'_{\mu\nu}$ and $e'_{\mu\nu}$ to try to compensate, giving
rise to numerical artefacts (note that the actual direct and indirect
correlation functions, $c_{\mu\nu}$ and $e_{\mu\nu}$, are
non-vanishing and finite inside the hard core region, thus the
divergence in $\beta v\lr$ unavoidably contaminates the offset
functions).  Numerically much better is to chose the long range
potential such that it is bounded inside the hard cores.  By far
the simplest choice is a truncated Coulomb law,
%
\begin{equation}
  v\lr=\Bigl\{\begin{array}{ll}
  \lB/\sigma & r\le\sigma\,,\\
  \lB/r & r > \sigma\,.
  \end{array}
\end{equation}
%
This does least damage to the potential outside the hard core.  With
this choice, the Fourier transform is easily shown to be
%
\begin{equation}
  {\tilde v}\lr=\frac{4\pi\lB}{k^2}\times\frac{\sin k\sigma}{k\sigma}\,.
\end{equation}
%
The RPM potential in the code is implemented using these choices.

As in the softened URPM, there are two ways the softened RPM can be
implemented.  The first way is to work purely in reciprocal space
which takes the model away from the SYM condition expressed in
\Eqref{eq:sp}.  In this approach the short range part $v'_{12}=0$
(outside the hard cores), and the long range part is
%
\begin{equation}
v_{12}\lr=-\frac{\lB\kT}{r} \erf(\kappa r)\,.
\end{equation}
%
The potential in reciprocal space and the derivatives follow
\mutmut\ from the softened URPM case (\ie\ URPM $\sigma\to0$ and
$\sigma'\to1/(2\kappa)$).  Unlike the na\"\i{}ve $\lB/r$ choice for
the RPM, this softened potential is well-behaved inside the hard core.
Thus the softened RPM is implemented with the truncated repulsive
Coulomb interaction between like charges, and the softened attractive
Coulomb interaction between unlike charges.
  
In the alternative approach we retain the SYM condition so that the
long range part is given by the vanilla RPM above, and the short range
part is
%
\begin{equation}
  v'_{12}\equiv-\Delta v_{12}=\frac{\lB\kT}{r}\erfc(\kappa r)
  +\Bigl(\frac{\lB\kT}{\sigma}-\frac{\lB\kT}{r}\Bigr)\Theta(\sigma-r)
\label{eq:dv12a}
\end{equation}
%
where $\Theta(\sigma-r)$ is the Heaviside function (and obviously,
$v'_{11}=v'_{22}=0$).  As in the URPM, we define $\Delta v_{12}$ as
the potential that should be \emph{added} to the softened RPM, to
recover the standard RPM.  This is so that we can use the softened
version as a reference fluid.

For both routes the mean field contributions to the virial route
pressure and energy are non-zero only for the unlike pairs.  We can
evaluate them with $v'_{12}$ given by \Eqref{eq:dv12}
%
\begin{eqnarray}
\displaystyle
-\frac{2\pi}{3}\int_\sigma^\infty\!\!dr\,r^3\,
\frac{\partial(\beta
  v'_{12})}{\partial r}=
\pi\lB\Bigl(\frac{\sigma\exp[-\kappa^2\sigma^2]}{\kappa\sqrt{\pi}}
+\Bigl(\frac{1}{2\kappa^2}-\frac{\sigma^2}{3}\Bigr) \erfc(\kappa
\sigma)\Bigr)\,.\\[9pt]
\displaystyle
2\pi \int_\sigma^\infty\!\!dr\,r^3\,\beta v'_{12}=
\pi\lB\Bigl(\frac{\sigma\exp[-\kappa^2\sigma^2]}{\kappa\sqrt{\pi}}
+\Bigl(\frac{1}{2\kappa^2}-\sigma^2\Bigr) \erfc(\kappa
\sigma)\Bigr)\,.
\end{eqnarray}
%
In the limit $\sigma\to0$ these both become $\pi\lB/2\kappa^2$
(\cf\ softened URPM).  In the limit $\kappa\to\infty$ (RPM case) both
vanish.

If we take the first, purely reciprocal space route, the failure to
satisfy the SYM condition means the compressibility and chemical
potentials need to take account of $v_{\mu\nu}\lr$.  This leads to the
long range contributions 
%
\begin{equation}
{4\pi}\int_0^\infty\!\!dr\,r^2\,\beta
v'_{12}=\frac{\pi\lB}{\kappa^2}\,.
\end{equation}
%
These should not be included if we follow the second route, and
vanishes in any case for the RPM case ($\kappa\to\infty$).

\section{Implementation notes}
%
Most of the code is self-expanatory, given the above mathematical
background.  Correlation functions in the real space domain are
discretised in an array of size $i = 1 \dots n-1$ with a spacing
$\delta_r$ so that $r_i = \delta_r \times i$.  Usually one choses
$\delta_r$ to be a small fraction of the relevant length scale
(\eg\ $\delta_r = 0.01\,\rc$) and $n\delta_r$ to be some large multiple
of the same (\eg\ $n\delta_r\approx 40\,\rc$).  The actual value of $n$
can be chosen to optimise the fast Fourier transform algorithm, for
instance $n=4096$.  The concomitant functions in the reciprocal space
domain are discretised in an array of size $j=1\dots n-1$ with a
spacing $\delta_k \equiv \pi / (n\delta_r)$ so that $k_j = \delta_k
\times j$.  The array size $n-1$ and the value $\delta_k$ are fixed by
the demands of the fast Fourier transform algorithm.

The implementation of the Fourier transforms is as follows.  A
discrete version of \Eqref{eq:fFB} is
%
\begin{equation}
{\tilde h}_j = \frac{2\pi\delta_r}{k_j} \times 
2\, {\textstyle\sum_{i=1}^{n-1}} r_i h_i \sin(\pi i j / n)
\end{equation}
%
where $j = 1\dots n-1$.  The quantity $2\,\sum_{i=1}^{n-1} r_i h_i
\sin(\pi i j / n)$ is computed by calling the \FFTW\ routine
\verb+RODFT00+ on $r_i h_i$.  From the \FFTW\ documenation, this
specific routine works best when the array length is of the form
$2^a3^b5^c7^d11^e13^f-1$ where $e+f$ is either 0 or 1 and the other
exponents are arbitrary (this means $n$ should be of the form of the
first term since the array length is $n-1$).  For the back-transform
the corresponding discrete version of \Eqref{eq:bFB} is
%
\begin{equation}
h_i = \frac{\delta_k}{(2\pi)^2 r_i}  \times 
2\,{\textstyle \sum_{j=1}^{n-1}} k_j {\tilde h}_j \sin(\pi i j / n)
\end{equation}
%
The quantity $2\,\sum_{j=1}^{n-1} k_j h_j \sin(\pi i j / n)$ is
computed by calling \verb+RODFT00+ on $k_j h_j$.

Version 1.7 of the code can handle an arbitrary number of components
(versions prior to this could only handle at most three components).
A species pair array index $(i, j)$ is mapped to a `function' index
$s$ according to the following rule
%
\begin{equation}
  s=\biggl\{\begin{array}{ll}
  i+j(j-1)/2 & \hbox{if $i\le j$\,,}\\[3pt]
  j+i(i-1)/2 & \hbox{if $i>j$\,.}
  \end{array}
  \label{eq:ij}
\end{equation}
%
This is implemented in various forms throughout the code.  The mapping
essentially labels the entries in the upper triangular form of a
symmetric matrix as shown here (where $i$ labels rows, and $j$ labels
columns)
%
\begin{equation}
  \begin{array}{c|ccccc}
    & 1 & 2 & 3 & 4 & \cdots\\
    \hline\\[-12pt]
    1 & 1 & 2 & 4 & 7 & \cdots\\
    2 &   & 3 & 5 & 8 & \cdots\\
    3 &   &   & 6 & 9 & \cdots\\
    4 &   &   &   & \zeroset{10} & \cdots\\
    \cdots & & & & & \cdots
  \end{array}\label{eq:mat}
\end{equation}
%
If necessary, one can recover the array indices $(i, j)$ from the the
index $s$ by the following rule, where $\lfloor\cdots\rfloor$ is the
`floor' function,
%
\begin{equation}
  j = \Bigl\lfloor{\frac{1+\sqrt{8s-7}}{2}}\Bigr\rfloor\,,\quad
  i = n - {j(j-1)}/{2}\,.
\end{equation}
%
This is not used in the code though.  The proof is left as an exercise.

The matrix operations in \Eqsref{eq:oznd} and \eqref{eq:ozndb} are
implemented using the standard \verb+MATMUL+ function in \FORTRAN\ 90.
The matrix inversions are not implemented \perse, but are rather done
by solving the matrix equation $A\cdot X=B$ using a bespoke subroutine
\verb+axeqb_reduce+ in the code which implements a simplified version
of Gauss-Jordan elimination with pivoting (but not $LDU$
factorisation) \cite{NR92}.

The Ng acceleration scheme is implemented by keeping track of a number
of previous iterations (typically 6), recycling the storage.  Thus the
main functions $c_{\mu\nu}'$ and $e_{\mu\nu}'$ are multidimensional:
the first axis is the spatial discretisation, the second axis is the
species pair index, and the third axis is the Ng trajectory index
(cyclical history array).  The Ng scheme involves also involves
solving $A\cdot X=B$ but this is done with a call to \verb+DSYSV+ in
the \LAPACK\ library (which does implement $LDU$ factorisation).  Now
you might think you could use \verb+DSYSV+ for the matrix operations
in the OZ inversions, or \verb+axeqb_reduce+ in the Ng accelerator,
but for some reason connected to the $LDU$ factorisation this leads to
numerical inaccuracies.  If anyone can shed any light on this please
let me know!

The thermodynamic quantities in section \ref{sec:thermo} are all
evaluated by standard trapezium rules, applied to the integrals given
therein.

If scripting with \python\ note that the \python\ array index is one
less than the \FORTRAN\ array index.  This conversion is seamlessly
and invisibly implemented in \verb+f2py+.  Also note that the optional
parameters in \FORTRAN\ function calls can be omitted also in \python,
and the code will gracefully fall back to the documented defaults.

\section{Usage notes}
%
The code is presented as a suite of functions in a \FORTRAN\ 90
module.  Thus it typically needs a driver code, which can be written
in \FORTRAN\ 90 or in \python\ with the use of \verb+f2py+ (see
examples below).  Thermodynamic integration schemes, for example the
compressibility route to the equation of state, are not implemented
here.  Rather the philosophy has been to make the solution of the
basic integral equation problem as fast and robust as possible.

\subsection{Routines}
%
\begin{itemize}
%
\item\verb+initialise+ -- Allocate all arrays given the number of
  components \verb+ncomp+, the number of real space points \verb+ng+,
  and the length of the Ng trajectory \verb+nps+.  Initialise the
  \FFTW\ plan for fast Fourier transforms.  This \verb+initialise+
  routine is typically called once, at the start, after the values of
  \verb+ncomp+, \verb+ng+, \verb+nps+, and \verb+deltar+ have been
  fixed.
%
\item\verb+dpd_potential(charge_type)+ -- Sets up the DPD potential
  described in section \ref{sec:dpd}.  This routine can be called
  multiple times.  The optional integer parameter indicates whether
  the potential should be initialised with Gaussian charges
  (\verb+1+), Bessel charges (\verb+2+), linear (Groot) charges
  (\verb+3+), Slater charges with exact interaction (\verb+4+), or
  Slater charges with approximate interaction (\verb+5+).  The linear
  case (\verb+3+) is currently incomplete in the sense that the
  thermodynamics is not calculated correctly.  If omitted, the
  \verb+charge_type+ defaults to Gaussian (\verb+1+).
%
\item\verb+urpm_potential(use_ushort)+ -- Sets up the ultrasoft
  restricted primitive model (URPM) potential with optional softening
  as described in section \ref{sec:softURPM}.  This routine can be
  called multiple times.  It checks that \verb+ncomp = 2+ and forces
  $z_1=1$ and $z_2=-1$.  The optional logical parameter
  \verb+use_ushort+ (default \verb+.false.+) indicates whether the
  potential should be formulated to contain a short range real space
  correction.
%
\item\verb+rpm_potential(use_ushort)+ -- Sets up the restricted
  primitive model (RPM) with optional softening as described in
  section \ref{sec:softRPM}.  This routine can be called multiple
  times.  It checks that \verb+ncomp = 2+ and forces $z_1=1$ and
  $z_2=-1$.  The optional logical parameter works the same way as in
  the URPM potential.
%
\item\verb+hs_potential+ -- Sets up one component hard spheres with
  diameter \verb+sigma+.  This routine can be called multiple times.
  It checks that \verb+ncomp = 1+.
%
\item\verb+hnc_solve+ -- Calculate the HNC solution.  Specifically,
  initialise $c'_{\mu\nu}$ if the logical flag \verb+cold_start+ is
  set, take enough Picard steps to pump-prime the Ng acceleration
  method, then attempt to converge to a solution.  This is the basic
  HNC solver routine and contains appropriate calls to
  \verb+oz_solve+, \verb+hnc_picard+, \verb+hnc_ng+ and
  \verb+conv_test+ below.  A warning is issued if the convergence
  criterion is not met.  If the logical flag \verb+auto_fns+ is also
  set (the default) the three \verb+make_*+ functions are also called,
  resulting in a complete solution to the problem.  The function
  \verb+hnc_solve+ should be called only after \verb+initialise+ and
  the potential has been initialised.  It may be called multiple times
  with different densities for a given potential, or the potential can
  be reset between calls.  In later calls, the existing $c'_{\mu\nu}$
  is used as a starting point unless the logical flag \verb+cold_start+ is
  reset.
%
\item\verb+msa_solve+ -- Calculate the MSA solution, using a similar
  iterative algorithm to the HNC closure. This routine is the basic
  MSA solver and contains appropriate calls to \verb+oz_solve+,
  \verb+msa_picard+, \verb+msa_ng+ and \verb+conv_test+.  A warning is
  issued if the convergence criterion is not met.  If the logical flag
  \verb+auto_fns+ is set (the default) the three \verb+make_*+
  functions are also called, resulting in a complete solution to the
  problem.  Like the HNC, the function \verb+msa_solve+ should be
  called only after \verb+initialise+ and the potential has been
  initialised.  It may be called multiple times with different
  densities for a given potential, or the potential can be reset
  between calls.  In these subsequent calls, the existing
  $c'_{\mu\nu}$ within the hard core regions is used as a starting
  point unless the logical flag \verb+cold_start+ is reset. In all
  cases though, $c'_{\mu\nu}$ outwith the hard core regions is set
  equal to $-\beta v'_{\mu\nu}$ for the current potential. Note that
  the MSA is the same as the RPA if there are no hard cores
  (therefore, use \verb+rpa_solve+ in that case!).
%
\item\verb+rpa_solve+ -- Calculate the RPA solution.  This involves
  only a single round trip through \verb+oz_solve+ from $c'_{\mu\nu} =
  -\beta v'_{\mu\nu}$.  Provided the logical flag \verb+auto_fns+ is
  set (the default), the three \verb+make_*+ functions are also
  called. The function \verb+rpa_solve+ should be called only after
  \verb+initialise+ and the potential has been initialised.  It may be
  called multiple times with different densities for a given
  potential, or the potential can be reset between calls.  The RPA
  closure cannot be used with hard cores (see section \ref{sec:RPA}).
%
\item\verb+save_reference+ -- In the presence of hard cores, EXP
  requires a reference state $\href_{\mu\nu}$, and this routine saves
  the current $h_{\mu\nu}$ solution.  Usually, this reference state
  will be the MSA solution to hard sphere problem without the tail
  potential, however this is not enforced.  See next entry for a note
  on the usage.
%
\item\verb+exp_refine+ -- Implement the EXP refinement.  Use
  \Eqref{eq:exp} to compute a refined $h_{\mu\nu}$ from the current
  solution and the reference $\href_{\mu\nu}$, then call
  \verb+oz_solve2+ to rebuild the complete solution according to
  \Eqref{eq:ozndb}.  Provided the logical flag \verb+auto_fns+ is set
  (the default), the three \verb+make_*+ functions are also called.  A
  call to \verb+initialise+ sets $\href_{\mu\nu}=0$, thus
  \verb+exp_refine+ is suitable for immediate use with soft potentials
  without hard cores.  To use \verb+exp_refine+ for potentials with
  hard cores we need to calculate and save the reference state, for example
  in \FORTRAN\ 90:
%
\begin{verbatim}
  [ .. remove tail, eg lb = 0.0 ]
  call msa_solve
  call save_reference
  [ .. restore tail, eg lb > 0.0 ]
  call msa_solve
  call exp_refine
\end{verbatim}
%
Note that \verb+exp_refine+ restores the reference state after
completing so it can be called multiple times with the same reference
state.
%
\item\verb+oz_solve+ -- Solve the OZ relation in the form of
  \Eqref{eq:oznd} to find $e'_{\mu\nu}$ from $c'_{\mu\nu}$.  As a
  byproduct, this generates ${\tilde e}_{\mu\nu}\myprime$ and ${\tilde
    c}_{\mu\nu}\myprime$.
%
\item\verb+oz_solve2+ -- Solve the OZ relation in the form of
  \Eqref{eq:ozndb} to find $e'_{\mu\nu}$ and $c'_{\mu\nu}$ from the
  \emph{reference} state $\href_{\mu\nu}$.  As a byproduct generate
  ${\tilde e}_{\mu\nu}\myprime$ and ${\tilde c}_{\mu\nu}\myprime$.
  Store the result in the first position in the cyclical history
  array.
%
\item\verb+hnc_picard+ -- Solve the HNC closure to find $c'_{\mu\nu}$
  from $e'_{\mu\nu}$, and take a Picard step in the sense of mixing a
  fraction \verb+alpha+ of the new $c'_{\mu\nu}$ with the old
  $c'_{\mu\nu}$.
%
\item\verb+hnc_ng+ -- Solve the HNC closure to find a new
$c'_{\mu\nu}$ given $e'_{\mu\nu}$, but using the Ng acceleration scheme.
%
\item\verb+msa_picard+ -- Solve the MSA closure to find $c'_{\mu\nu}$
  from $e'_{\mu\nu}$, and take a Picard step in the sense of mixing a
  fraction \verb+alpha+ of the new $c'_{\mu\nu}$ with the old
  $c'_{\mu\nu}$.
%
\item\verb+msa_ng+ -- Solve the MSA closure to find a new
$c'_{\mu\nu}$ given $e'_{\mu\nu}$, but using the Ng acceleration scheme.
%
\item\verb+conv_test+ -- Check convergence in iterative schemes by
  calculating the difference (saved in \verb+error+) between the
  current and immediately preceeding $c'_{\mu\nu}$.
%
\item\verb+make_pair_functions+ -- from \Eqref{eq:hrs}.  The
  choice to present $h_{\mu\nu}$ rather $g_{\mu\nu}$ is made because
  there may be interest in the asymptotic behaviour of
  $h_{\mu\nu}\to0$ as $r\to\infty$, and it is assumed the user can add
  `1' to get the pair distribution functions.
%
\item\verb+make_structure_factors+ -- from \Eqref{eq:skdef}.
%
\item\verb+make_thermodynamics+ -- everything in section
  \ref{sec:thermo}.
%
\item\verb+write_params+ -- Write out basic information.  Often useful
  as a check that the right potential is being solved.
%
\item\verb+write_thermodynamics+ -- Write out all thermodynamic
  quantities, assuming \verb+make_thermodynamics+ has been called.
%
\end{itemize}
%
Note that the three \verb+make_*+ routines are independent and can be
called in any order.

\subsection{User parameters}
%
\begin{itemize}
%
\item\verb+ncomp+ -- The number of component species, $\text{arbitrary}\ge
  1$ (default 1).
%
\item\verb+ng+ -- The real space grid size, $n$, usually a power of 2
  (default 4096).
%
\item\verb+nps+ -- The length of the Ng trajectory, equivalently the
  size of the cyclical history array (usually safe to leave at the
  default 6).
%
\item\verb+npic+ -- The number of Picard steps to pump-prime the Ng
  acceleration scheme (usually safe to leave at the default 6, but
  should be $\ge$ \verb+nps+).
%
\item\verb+maxsteps+ -- maximum number of steps to take before giving up on
convergence (usually safe to leave at default 100).
%
\item\verb+verbose+ -- Logical flag to print solver diagnostics (default
  \verb+.false.+).
%
\item\verb+silent+ -- Suppress printing of warning/error messages
  (default \verb+.true.+).
%
\item\verb+cold_start+ -- Logical flag to indicate whether the main
  solvers should execute a cold start by initialising $c'_{\mu\nu}$.
  This flag is initially set \verb+.true.+, then set \verb+.false.+
  after a main solver call since it is often the case that a
  subsequent problem can re-use the current solution as a good initial
  guess.  The flag can of course be reset \verb+.true.+ by the user at
  any point.
%
\item\verb+start_type+ -- In a HNC cold start initialise
  $c'_{\mu\nu}=0$ (start type 1), $c'_{\mu\nu}=-\beta v'_{\mu\nu}$
  (start type 2), or $c'_{\mu\nu}=\exp(-\beta v'_{\mu\nu})-1$ (start
  type 3, default).  Usually safe to leave at default.
%
\item\verb+auto_fns+ -- Logical flag (default \verb+.true.+) whether
  to call the \verb+make_*+ routines on exit of the main solver
  routines.  Since these routines are relatively inexpensive one would
  usually leave this flag set.
%
\item\verb+deltar+ -- The real space grid spacing $\delta_r$ (default 0.01).
%
\item\verb+alpha+ -- Fraction of new solution in Picard method
  (usually safe to leave at default 0.2).
%
\item\verb+tol+ -- Error tolerance for claiming convergence (usually
  safe to leave at default $10^{-12}$).
%
\item\verb+rc+ -- DPD repulsion range $r_c$ (default 1.0).
%
\item\verb+lb+ -- Coulomb coupling strength $\lB$ (default 0.0)
%
\item\verb+sigma+ -- smearing length $\sigma$ (default 1.0) in the
  case of Gaussian or Bessel charges in the DPD potential and URPM
  potentials; hard core diameter $\sigma$ in the case of the RPM or
  hard sphere potentials.
%
\item\verb+kappa+ -- parameter $\kappa$ (default $-$1.0) in the
  softened RPM potential.  A negative value is interpreted as
  $\kappa\to\infty$ and generates the vanilla RPM.
%
\item\verb+sigmap+ -- charge size $\sigma'$ (default 1.0), used for
  the softened URPM potential.
%
\item\verb+rgroot+ -- linear (Groot) charge smearing range $R$
  (default 1.0) in the DPD potential (\verb+charge_type = 3+).
%
\item\verb+lbda+ -- Slater charge smearing length $\lambda$ (default
  1.0) in the DPD potential with exact interaction
  (\verb+charge_type = 4+).  The name is truncated because
  \verb+lambda+ is a reserved word in python.
%
\item\verb+beta+ -- Slater charge smearing parameter $\beta$
  (default 1.0) in the DPD potential with approximate interaction
  (\verb+charge_type = 5+).  One would usually calculate this from
  $\beta=1/\lambda$, or (better) $\beta=5/(8\lambda)$.
%
\item\verb+rho(:)+ -- An array of size \verb+ncomp+ where
  \verb+rho(mu)+ is the density $\rho_\mu$.  All entries default to
  zero after a call to \verb+initialise+.
%
\item\verb+z(:)+ -- An array of size \verb+ncomp+ where
  \verb+z(mu)+ is the valence $z_\mu$.  All entries default to
  zero after a call to \verb+initialise+.
%
\item\verb+diam(:)+ -- An array of size \verb+ncomp+ where
  \verb+diam(mu)+ is the hard core diameter $d_\mu$.  All entries
  default to zero after a call to \verb+initialise+.
%
\item\verb+arep(:, :)+ -- An array of size
  \verb+ncomp+$\times$\verb+ncomp+ where \verb+arep(mu, nu)+ is
  $A_{\mu\nu}$.  Note that only elements above the diagonal are used,
  \eg\ \verb+arep(1,2)+, \verb+arep(1,3)+, and \verb+arep(2,3)+.  {\bf
    Elements below the diagonal are ignored}, \cf\ the matrix in
  \Eqref{eq:mat}.  All entries default to zero after a call to
  \verb+initialise+.
%
\end{itemize}

\subsection{Outputs}
%
\begin{itemize}
%
\item\verb+deltak+ -- The reciprocal space grid spacing
  $\delta_k=\pi/(n\delta_k)$, fixed by the needs of the fast Fourier
  transform algorithm.  Available after a call to \verb+initialise+.
%
\item\verb+r(:)+ -- The array \verb+r(i)+ contains
  $r_i=i\times\delta_r$ for $i=1\dots n-1$.  Available after a call to
  \verb+initialise+.
%
\item\verb+k(:)+ -- The array \verb+k(j)+ contains
  $k_j=j\times\delta_k$ for $j=1\dots n-1$.  Available after a call to
  \verb+initialise+.
%
\item\verb+hr(:, :, :)+ -- An array containing the total correlation
  functions $h_{\mu\nu}(r)$ as in \Eqref{eq:hrs}.  Specifically
  \verb+hr(i, mu, nu)+ contains $h_{\mu\nu}(r_i)$ where
  $r_i=i\times\delta_r$ for $i=1\dots n-1$.  The values at $r=0$ can
  be obtained by linear extrapolation \cite{KMP04}.  These arrays are
  available after a call to \verb+make_pair_functions+ which is done
  automatically by the solver routines unless \verb+auto_fns+ is
  turned off.
%
\item\verb+sk(:, :, :)+ -- An array containing structure factors
  $S_{\mu\nu}(k)$ as in \Eqref{eq:skdef}.  Specifically
  \verb+sk(j, mu, nu)+ contains $S_{\mu\nu}(k_j)$ where
  $k_j=j\times\delta_k$ for $j=1\dots n-1$. These arrays are available
  after a call to \verb+make_structure_functions+ which is done
  automatically by the solver routines unless \verb+auto_fns+ is
  turned off.
%
\item\verb+error+ -- The final convergence error estimate for the
  iterative solvers, should be less than \verb+tol+ if converged.
%
\item\verb+return_code+ -- Integer indicates success if
  \verb+return_code = 0+, otherwise an error occurred (see next).
%
\item\verb+error_msg+ -- String (character array) describing the
  problem in the case that \verb+return_code+ is non-zero.  The
  message is printed out anyway unless the \verb+silent+ flag is set.
  Note that in \python\ a returned string appears as fixed
  length binary array and needs massaging to get into a printable
  form, thus for example:
%
\begin{verbatim}
#!/usr/bin/env python3
from oz import wizard as w
...
if w.return_code: 
    print('Error: %s' % str(w.error_msg, 'utf-8').strip())
    exit()
\end{verbatim}
%
\item\verb+closure_name+ -- String (character array) giving the
  last-used closure (a three-letter acronym).
%
\item\verb+potential_name+ -- String (character array) describing the
  potential model last used to set the potential arrays.
%
\end{itemize}

The following quantities are available after a call to
\verb+make_thermodynamics+ (which is done automatically unless
\verb+auto_fns+ is turned off).  They are also reported by
\verb+write_thermodynamics+.
%
\begin{itemize}
%
\item\verb+press+ -- The virial route pressure $\beta p$.
%
\item\verb+cf_mf+ -- The mean-field contribution to the
virial route compressibility factor $\beta p/\rho$.
%
\item\verb+cf_gc+ -- The contact contribution to the
virial route compressibility factor $\beta p/\rho$.
%
\item\verb+cf_xc+ -- The correlation contribution to the
 virial route compressibility factor $\beta p/\rho$.
%
\item\verb+comp+ -- The compressibility $\partial(\beta
  p)/\partial\rho$ (not to be confused with the 
  compressibility factor $\beta p/\rho$).
%
\item\verb+comp_xc+ -- The correlation contribution to the
  compressibility.
%
\item\verb+uv+ -- Internal energy density $\beta U/V$ (excludes
  kinetic contribution).
%
\item\verb+un+ -- Internal energy per particle $\beta U/N$ (excludes
  kinetic contribution).
%
\item\verb+un_mf+ -- The mean field contribution to \verb+un+.
%
\item\verb+un_corrn+ -- The correlation contribution to \verb+un+.
%
\item\verb+muex(:)+ -- Chemical potentials, \verb+muex(mu)+ contains
  $\beta\muex_\mu$ (HNC only).
%
\item\verb+fvex+ -- Excess free energy density $\beta \Fex/V$ (HNC only).
%
\item\verb+fnex+ -- Excess free energy per particle $\beta \Fex/N$ (HNC only).
%
\end{itemize}

The following additional variables may be useful.
%
\begin{itemize}
%
\item\verb+c(:, :, 1)+ and \verb+e(:, :, 1)+ -- After a successful
  call to one of the main solver routines, these contain the most
  converged solution, such that for example \verb+c(i, s, 1)+ contains
  $c'_{\mu\nu}(r_i)$, where \verb+s+ is the species pair index as in
  \Eqref{eq:ij}.  Remember to subtract $\beta v\lr_{\mu\nu}$ if
  comparing these functions to literature results for $c_{\mu\nu}$.
%
\item\verb+ck(:, :)+ and \verb+ek(:, :)+ -- Arrays containing the
  Fourier-transformed offset direct and indirect correlation
  functions, such that for example \verb+ck(j, s)+ contains ${\tilde
    c}_{\mu\nu}\myprime(k_j)$ where \verb+s+ is the species pair
  index.
%
\item\verb+h0(:, :)+ -- An array containing the reference state total
  correlation function, such that \verb+h0(i, s)+ contains
  $\href_{\mu\nu}(r_i)$ where \verb+s+ is the species pair index. Used
  by \verb+exp_refine+ and \verb+oz_solve2+.
%
\item\verb+ushort(:, :)+ and \verb+ulong(:, :)+ -- Arrays containing
  the potential, such that \verb+ushort(i, s)+ contains $\beta
  v'_{\mu\nu}(r_i)$ and \verb+ulong(i, s)+ contains $\beta
  v\lr_{\mu\nu}(r_i)$, where \verb+s+ is the species index.
%
\item\verb+expnegus(:, :)+ -- A similar array containing $\exp(-\beta
  v'_{\mu\nu})$.
%
\item\verb+dushort(:, :)+ and \verb+dulong(:, :)+ -- Similar arrays
  containing the derivatives of the potential, $\partial(\beta
  v'_{\mu\nu})/\partial r$ and $\partial(\beta v\lr_{\mu\nu})/\partial
  r$.
%
\item\verb+ulongk(:, :)+ -- Array containing the Fourier transform of
  the long range part of the potential, such that \verb+ulongk(j, s)+
  contains $\beta{\tilde v}\lr_{\mu\nu}(k_j)$, where \verb+s+ is the
  species index.
%
\item The fixed parameters \verb+pi+, \verb+twopi+, and \verb+fourpi+ 
($\pi$, $2\pi$, and $4\pi$) are available and can be utilised if
convenient.
%
\end{itemize}

Of these, the potential parts may of course be initialised in the
driver routine, as may the reference state total correlation function
in which latter case \verb+oz_solve2+ acts as a general OZ solver.
Note that the HNC, MSA and RPA solvers require \emph{only} the long
range potential in reciprocal space, $\beta{\tilde v}\lr_{\mu\nu}$
(\verb+ulongk+), and the exponentiated short range potential,
$\exp(-\beta v'_{\mu\nu})$ (\verb+expnegus+), thus a very minimalist
approach can be taken to specifying the potential.  The EXP solver
uses in addition the long range potential in real space, $\beta
v\lr_{\mu\nu}$ (\verb+ulong+).  The full suite of thermodynamics
calculations do require all the remaining potential functions though.

\section{Examples}
%
\subsection{Virial route pressure}
\label{sec:vrp}
%
As a basic example the following minimalist code solves the
HNC closure problem for standard one-component DPD at $\rho=3$ and
$A=25$, and prints the virial route pressure,
%
\begin{verbatim}
#!/usr/bin/env python3
from oz import wizard as w
w.initialise()
w.arep[0,0] = A = 25.0
w.dpd_potential()
w.rho[0] = rho = 3.0
w.hnc_solve()
print('rho =', rho, ' A =', A)
print('pressure =', w.press)
print('energy density =', w.uv)
\end{verbatim}
%
with the result
%
\begin{verbatim}
rho = 3.0  A = 25.0
pressure = 23.564147563638308
energy density = 13.761952448736555
\end{verbatim}
%
The first line imports the \verb+wizard+ of \verb+oz+ and renames it
\verb+w+.  The \verb+oz+ package and the \verb+wizard+ module are
automatically generated by \verb+f2py+.  The actual results from
accurate Monte-Carlo simulations are $p=23.71\pm0.01$ and
$e=13.83\pm0.01$, so the HNC virial route result is in error by
$\approx0.5$\%.

\begin{figure}
\begin{center}
\includegraphics[width=2.5in]{gofr.png}~~~%
\includegraphics[width=2.5in]{sofk.png}
\end{center}
\caption{HNC pair distribution function and structure factor for
  standard DPD at $\rho=3$ and $A=25$.\label{fig:gs}}
\end{figure}

A virtue of \python\ is the large package library, for instance for
graphical output.  The following code uses the \verb+matplotlib+
package to plot the pair distribution function and structure factor
(with a standard normalisation),
%
\begin{verbatim}
#!/usr/bin/env python3
import matplotlib.pyplot as plt
from oz import wizard as w
w.initialise()
w.arep[0,0] = A = 25.0
w.dpd_potential()
w.rho[0] = rho = 3.0
w.hnc_solve()
plt.figure(1)  # This will be g(r)
imax = int(3.0 / w.deltar)
plt.plot(w.r[0:imax], 1.0 + w.hr[0:imax,0,0])
plt.xlabel('$r$')
plt.ylabel('$g(r)$')
plt.figure(2)  # This will be S(k)
jmax = int(25.0 / w.deltak)
plt.plot(w.k[0:jmax], w.sk[0:jmax,0,0]/rho)
plt.xlabel('$k$')
plt.ylabel('$S(k)$')
plt.show()
\end{verbatim}
%
The result is shown in Fig.~\ref{fig:gs} (imagine trying to do this in
pure \FORTRAN!).

\subsection{Compressibility route pressure}
%
The following \python\ routine integrates the compressibility along an
isotherm to calculate the compressibility route pressure,
%
\begin{verbatim}
def cr_press(drho):
    p_xc = prev = 0.0
    n = int(w.rho[0]/drho + 0.5)
    w.cold_start = 1
    for i in range(n):
        w.rho[0] = drho * (i + 1.0)
        w.hnc_solve()
        p_xc = p_xc + 0.5 * drho * (prev + w.comp_xc)
        prev = w.comp_xc
    return w.rho[0] + p_xc
\end{verbatim}
%
Points to note are that a trapezium rule is used for the numerical
integration, and the routine actually integrates the excess
compressibility since the ideal contribution to the pressure is
trivially $\rho$.  When the \verb+cr_press+ routine finishes, the
system is in the same state as it started, with the thermodynamics
completely solved.  So for example
%
\begin{verbatim}
w.initialise()
w.arep[0,0] = 25.0
w.dpd_potential()
w.rho[0] = 3.0
print('CR pressure =', cr_press(0.05))
print('VR pressure =', w.press)
\end{verbatim}
%
results in
%
\begin{verbatim}
CR pressure = 22.6662332442
VR pressure = 23.5641475636383
\end{verbatim}
%
Pressures calculated by the two routes differ by about 4\%
which appears to be typical for HNC for soft potentials.  Also,
further investigation reveals the dependence on $\delta\rho$ is linear
and extrapolates to $p=22.31$ at $\delta\rho\to0$, for $\rho=3$ and
$A=25$.  So the error from discretising the isotherm is here $\alt2$\%.

\subsection{Free energy}
\label{sec:feng}
%
One can continue in the same vein, to calculate the pressure \via\ the
energy route, but this really involves the calculation of the free
energy.  This can be done by coupling constant integration.  The basic
result can be derived by differentiating the fundamental expression
$e^{-\beta F} = \int\!d\Omega\,e^{-\beta V}$ with respect to a
parameter in the potential $V$.  For instance for standard
single-component DPD, ${\partial F}/{\partial A}={\langle
  V\rangle}/{A}$ where $\langle V\rangle\equiv U$ is the internal
energy.  Thus
%
\begin{equation}
F=\int_0^A\frac{dA'}{A'}\,\langle V\rangle_{A'}
\label{eq:fint}
\end{equation}
%
which should be evaluated along an isochore (constant density line).
The following \python\ routine uses this to compute the excess free
energy per particle, $\fNex$,
%
\begin{verbatim}
def fnex1(dA):
    fnex_xc = prev = 0.0
    n = int(w.arep[0,0]/dA + 0.5)
    w.cold_start = 1
    for i in range(n):
        w.arep[0,0] = dA * (i + 1.0)
        w.dpd_potential(1)
        w.hnc_solve()
        curr = w.un_xc / w.arep[0,0]
        fnex_xc = fnex_xc + 0.5*dA*(prev + curr)
        prev = curr
    return w.un_mf + fnex_xc
\end{verbatim}
%
Points to note are again that this implements a trapezium rule, and
that the actual integration is carried out for the correlation
contribution for which $U/A$ vanishes as $A\to0$.  The mean field
contribution to the internal energy is strictly proportional to $A$,
so it passes unscathed through \Eqref{eq:fint} to contribute
to the free energy.

For the HNC specifically, we could also calculate $\fNex$ by integrating
$\muex$ along an isotherm, giving a second routine
%
\begin{verbatim}
def fnex2(drho):
    fvex = prev = 0.0
    n = int(w.rho[0]/drho + 0.5)
    w.cold_start = 1
    for i in range(n):
        w.rho[0] = drho * (i + 1.0)
        w.hnc_solve()
        fvex = fvex + 0.5*drho*(prev + w.muex[0])
        prev = w.muex[0]
    return fvex / w.rho[0]
\end{verbatim}
%
Note that integrating $\muex$ with respect to $\rho$ generates the
excess free energy density, so the routine divides this $\rho$ to get
$\fNex$.  Both routines leave the system in the same state as it
started.  As an example, to test these,
%
\begin{verbatim}
w.initialise()
w.rho[0] = 3.0
w.arep[0,0] = 25.0
w.dpd_potential()
print('energy fnex =', fnex1(0.1))
print('mu fnex =    ', fnex2(0.05))
print('HNC fnex =   ', w.fnex)
\end{verbatim}
%
(the last of these uses the built-in free energy evaluation for HNC), giving
%
\begin{verbatim}
energy fnex = 5.31588405588
mu fnex =     5.31606130857
HNC fnex =    5.3159336139428195
\end{verbatim}
%
We see that all three routes agree to the third decimal place, which
is a good test of convergence.  Sections \ref{sec:vrp}--\ref{sec:feng}
have been coded up as \verb+examples.py+.

\subsection{Further examples}

\begin{itemize}
%
\item\verb+gw_p_compare.py+ -- Plot $(p-\rho) /A\rho^2$ versus $\rho$,
  and compare with data from Fig.~4 of \Refcite{GW97}.  The agreement
  is very good.
%
\item\verb+wsg_fig5_compare.py+ -- The free energy difference $\Delta
  F$ between a system of $N-1$ A particles and a B particle, and a
  pure system of $N$ A particles (as a function of $\Delta A$, where
  $A_{00}=A_{11}=25$, $A_{01}=25+\Delta A$), was introduced in
  \Refcite{WSG01} and is representative of an effective $\chi$ value.
  Here, the free energy change is calculated as $\Delta
  F=\muex_1-\muex_0$ in a two component problem with the mole fraction
  of the second species set to zero ($\rho_0=3$ and $\rho_1=0$).  The
  results are compared with data from Fig.~5 of \Refcite{WSG01}, and
  with separate accurate Monte-Carlo simulations using trial particle
  identity swap moves. 
%
\item\verb+x_dmu_compare.py+ -- In an unpublished extension to
  \Refcite{WSG01}, $\Delta F=\muex_1-\muex_0$ is plotted as a function
  of the mole fraction $x$ in a mixture where $\rho_0=(1-x)\rho$ and
  $\rho_1=x\rho$, for $\Delta A=5$ (one cannot have $\Delta A$ too
  large otherwise an underlying demixing transition causes HNC to
  fail).  The results are compared to accurate Monte-Carlo (MC)
  simulations.  The agreement between HNC and MC is excellent,
  including the small deviation from linearity.  The near-exact linear
  dependence can be represented by $\Delta F = \chi (1 - 2x)$ where
  $\chi$ is the Flory $\chi$-parameter.  This is the basic reason why
  DPD is so good at representing fluid mixtures which fit
  Flory-Huggins (regular solution) theory.
%
\item\verb+hm_fig4-2_compare.py+ solves for the virial and
  compressibility route pressure for the MSA and HNC, for hard
  spheres, and compares with the accurate Carnahan-Starling equation of
  state, as in Fig 4.2 of \Refcite{HM06}.  See comments in file for
  more details.
%
\item\verb+hm_fig10-2_compare.py+ solves the MSA and HNC for the
  RPM and compares with Monte-Carlo data, as in Fig 10.2 of
  \Refcite{HM06}.  See comments in file for more details.
%
\item\verb+urpm_oneoff.py+ and \verb+urpm_scan.py+ are a pair of
  driver routines for solving the ultrasoft restricted primitive model
  (URPM) \cite{WVA+13}, including the possibility of a neutral solvent
  species.  The first routine can be used for one-off calculations of
  structure and thermodynamics at a chosen state point.  The second
  routine is intended to scan a range of densities to zero in on the
  Kirkwood transition between pure exponential and damped oscillatory
  behaviour in the tails of the total correlation functions.  Both
  routines are driven by command line options, via the \verb+argparse+
  module.  To see the available options use \verb+--help+.  The option
  \verb+--ncomp+ selects between the pure URPM case (\verb+--ncomp=2+,
  default) and the solvated URPM case (\verb+--ncomp=3+).
%
\item\verb+rpm_oneoff.py+ is a similar driver routine for exploring
  the RPM.
%
\item\verb+hs_oneoff.py+ is a similar driver routine for exploring
  hard spheres.
%
\item\verb+urpm_press.py+ and \verb+urpm_targp.py+ are auxiliary
  scripts targetted at the high density URPM phase boundary where HNC
  still has a solution even at $\lB\gtrsim100$.
%
\item\verb+surpm_oneoff.py+ and \verb+surpm_gr.py+ report various
  aspects of solutions for the softened URPM model. 
%
\item\verb+wertheim_thermo.py+, \verb+wertheim_solver.py+ are python
  scripts, and \verb+wertheim_coex.sh+ is a shell script, to solve the
  URPM coexistence problem using HNC for the high density phase
  boundary and Wertheim theory with a softened URPM reference fluid
  (solved by HNC) for the low density phase boundary.  For more
  details and sample results see comments in the scripts.
%
\item Sundry items: \verb+driver2-4.f90+ and \verb+urpm.f90+ are
  sample FORTRAN 90 driver routines, retained for regression testing,
  as is \verb+driver3.py+; and \verb+fftw_test.f90+ is test code to
  comprehend how FFTW works.
%
\end{itemize}

%\vspace{36pt}
\appendix
%
\section{Appendix}
%
\subsection{MSA charge-charge structure factor for RPM}
%
The MSA solution for the RPM is known analytically \cite{WL72}.  Some
results are summarised here, for benchmarking the code.  We
focus particularly on the charge-charge structure factor as the closed
form solution is fairly simple.  As in section \ref{sec:softRPM} we
suppose the potential is given by
%
\begin{equation}
  \beta v_{ij}=\Bigl\{\begin{array}{ll}
  \infty & r\le R\\[3pt]
  z_iz_j\lB/r & r>R
  \end{array}
\end{equation}
%
where $z_1=1$ and $z_2=-1$. In this section we shall use $R$ for the
hard core diameter, to match with literature notation.  We shall set
$\rho_1=\rho_2=\rho/2$.  Generically, because of symmetry
($h_{11}=h_{22}$ and $h_{12}=h_{21}$ \etc), the charge-charge
structure factor is given by
%
\begin{equation}
  S_{ZZ}=1+{\textstyle\frac{1}{2}}\rho(\tilde h_{11}-\tilde h_{12})\,.
\end{equation}
%
The symmetry also means that the OZ equation decouples,
%
\begin{equation}
  \tilde h_{11}-\tilde h_{12}=
  \tilde c_{11}-\tilde c_{12} + 
  {\textstyle\frac{1}{2}}\rho(\tilde h_{11}-\tilde h_{12})
  (\tilde c_{11}-\tilde c_{12})\,.
\end{equation}
%
This means that the charge-charge structure factor can be expressed
simply in terms of the Fourier-transformed direct correlation functions,
%
\begin{equation}
  S_{ZZ}=[1-{\textstyle\frac{1}{2}}\rho(\tilde c_{11}-\tilde c_{12})]^{-1}\,.
\end{equation}
%
Now we turn to the MSA solution of the RPM \cite{WL72}.  There are two
ways to calculate the charge-charge structure factor, both of which
lead to the same result.  The first is to Fourier transform the known
solution to the real space direct correlation function.  For this, we
introduce the inverse Debye length $\kappa=(4\pi\lB\rho)^{1/2}$, and
dimensionless version of the same, $x=\kappa R$.  In terms of the
latter we define $B=[1+x-(1+2x)^{1/2}]/x$.  Then the MSA solution for
the direct correlation function difference is
$c_{11}-c_{12}=-({4\lB}/{R})(B-{B^2 r}/{2R})$ for $r\le R$, and
$c_{11}-c_{12}= -{2\lB}/{r}$ for $r > R$ (the latter reflects the MSA
choice $c_{\mu\nu}=-\beta v_{\mu\nu}$ outside the hard core).  The
reason for focusing on the charge-charge structure factor is now
apparent as this difference expression does not involved the
Percus-Yevick solution for uncharged hard spheres, which would
otherwise accompany the individual direct correlation functions.

To do the Fourier transform we write this as
%
\begin{equation}
  \half(c_{11}-c_{12})=-\frac{\lB}{r}+\Bigl[\frac{\lB}{r}
    -\frac{2\lB}{R}\Bigl(B-\frac{B^2 r}{2R}\Bigr)\Bigr]\,\Theta(R-r)
  \label{eq:msa1}
\end{equation}
%
then, \cf\ \Eqref{eq:fFB},
%
\begin{equation}
  \half(\tilde c_{11}-\tilde c_{12})=-\frac{4\pi\lB}{k^2}+
  \frac{4\pi}{k}\int_0^R\!dr\,r\,\sin kr\,\Bigl[\frac{\lB}{r}
  -\frac{2\lB}{R}\Bigl(B-\frac{B^2 r}{2R}\Bigr)\Bigr]\,.
\end{equation}
%
The integral is easily evaluated and inserted in the expression
for the charge-charge structure factor to obtain, after some
simplification,
%
\begin{equation}
  S_{ZZ}(k)=\frac{k^4}{k^4+8q^4+4q^2(k^2-2q^2)\cos kR+8kq^3\sin kR}\,.
  \label{eq:szzrpm}
\end{equation}
%
In this, $q=[(1+2x)^{1/2}-1]/(2R)$.\footnote{Thus $x=2qR(1+qR)$, and
  $B=qR/(1+qR)$.}  \Eqref{eq:szzrpm} can be compared directly with the
charge-charge structure factor reported by \verb+rpm_oneoff.py+.
Complete agreement to within numerical accuracy should be observed.
Note that correctly $S_{ZZ}(k)=k^2/\kappa^2+O(k^4)$ as $k\to0$ (thus
the MSA satisfies the Stillinger-Lovett moment conditions), and
$S_{ZZ}(k)\to1$ as $k\to\infty$.

For completeness, the partner to \Eqref{eq:msa1} is the Percus-Yevick
hard sphere solution \cite{HM06, Wert63}
%
\begin{equation}
  \half(c_{11}+c_{12})=
  -[\lambda_1-6\eta\lambda_2r/R
  +\half\eta\lambda_1(r/R)^3]\,\Theta(1-x)
  \label{eq:msa2}
\end{equation}
%
where $\lambda_1=(1+2\eta)^2/(1-\eta)^4$,
$\lambda_2=(1+\half\eta)^2/(1-\eta)^4$, and $\eta=\pi\rho R^3/6$ is
the packing fraction.

An alternative route to the charge-charge structure factor uses
analytic continuation on the Laplace-transformed direct correlation
function.  Specifically, Waisman and Lebowitz \cite{WL72} define $\bar
G(s)=\int_0^\infty dr\,e^{-sr} r\, [g_{11}(r)-g_{12}(r)]$ and give the
result (see also Stell and Sun \cite{SS75})
%
\begin{equation}
  \bar G(s)=-s(2q^2/\pi\rho)[(s^2+2qs+2q^2)e^{sR}-2q^2]^{-1}\,.
\end{equation}
%
From the definitions it is easy to infer
%
\begin{equation}
  S_{ZZ}(k)=1+\frac{2\pi\rho}{k}\imaginary \bar G(-ik)\,.
\end{equation}
%
If the indicated operations are carried out, one indeed recovers
\Eqref{eq:szzrpm}.

\thebibliography{}

\bibitem{WVA+13} P.~B.~Warren, A.~Vlasov, L.~Anton and A.~J.~Masters
  ``Screening properties of Gaussian electrolyte models, with
  application to dissipative particle dynamics'',
  J. Chem. Phys. {\bf138}, 204907 (2013).

\bibitem{Ng74} K.-C. Ng, ``Hypernetted chain solutions for the
  classical one-component plasma up to $\Gamma=7000$'',
  J. Chem. Phys. {\bf61}, 2680--2689 (1974).

\bibitem{HM06} J.-P.~Hansen and I.~R.~McDonald, ``Theory of simple
  liquids 3rd edition'' (Academic Press, Amsterdam, 2006).

\bibitem{KMP04} C.~T.~Kelley and B.~Montgomery Pettitt, ``A fast
  solver for the Ornstein-Zernike equations'',
  J. Comp. Phys. {\bf197}, 491--501 (2004).  Beware typos in this
  paper!  Note, $i$ and $j$ in the main text are $i-1$ and $j-1$ in
  this paper, and $n = N-1$.  This paper also suggests that $c$ and
  $e$ at $r = 0$ are evaluated by simple linear extrapolation, $c_0 =
  2 c_1 - c_2$ and $e_0 = 2 e_1 - e_2$; and $c$ and $e$ at $r =
  n\delta_r$ are zero, $c_n = e_n = 0$.

\bibitem{Vrbka09} L.~Vrbka, M.~Lund, I.~Kalcher, J.~Dzubiella,
  R.~R.~Netz, and W.~Kunz, ``Ion-specific thermodynamics of
  multicomponent electrolytes: A hybrid HNC/MD approach'',
  J.\ Chem.\ Phys.\ {\bf131}, 154109 (2009).
  
\bibitem{WV14} P.~B.~Warren and A.~Vlasov ``Screening properties of
  four mesoscale smoothed charge models, with application to
  dissipative particle dynamics'', J.\ Chem.\ Phys.\ {\bf140}, 084904
  (2014).

\bibitem{Groot03}R.~D.~Groot, ``Electrostatic interactions in
  dissipative particle dynamics---simulation of polyelectrolytes and
  anionic surfactants'', J.\ Chem.\ Phys.\ {\bf118}, 11265 (2003).

\bibitem{GM+06}M.~Gonz\'alez-Melchor, E.~Mayoral, M. E. Vel\'azquez
  and J. Alejandre, ``Electrostatic interactions in dissipative
  particle dynamics using the Ewald sums'', J.\ Chem.\ Phys.\ {\bf125},
  224107 (2006).

\bibitem{NR92} W. H. Press, B. P. Flannery, S. A. Teukolsky,
  W. T. Vetterling, ``Numerical recipes in FORTRAN 77: 2nd edition''
  (CUP, Cambridge, 1992).

\bibitem{GW97} R.~D.~Groot and P.~B.~Warren, ``Dissipative particle
  dynamics: bridging the gap between atomistic and mesoscopic
  simulation'', J.\ Chem.\ Phys.\ {\bf107}, 4423 (1997).

\bibitem{WSG01} C.~M.~Wijmans, B.~Smit and R.~D.~Groot, ``Phase
  behavior of monomeric mixtures and polymer solutions with soft
  interaction potentials'', J.\ Chem.\ Phys.\ {\bf114}, 7644 (2001).

\bibitem{WL72} E.~Waisman and J.~L.~Lebowitz, ``Mean spherical model
  integral equation for charged hard spheres. II Results'',
  J.\ Chem.\ Phys.\ {\bf56}, 3093 (1972).

\bibitem{Wert63} M.~S.~Wertheim, ``Exact solution of the Percus-Yevick
  integral equation for hard spheres'', Phys.\ Rev.\ Lett.\ {\bf10},
  321 (1963).

\bibitem{SS75} G.~Stell and S.~F.~Sun, ``Generalised mean spherical
  approximation for charged hard spheres: The electrolyte regime'',
  J.\ Chem.\ Phys.\ {\bf63}, 5333 (1975).

\end{document}
